\documentclass[12pt]{extarticle}
\usepackage[paperwidth=8.5in, paperheight=11in, top=1in, bottom=1in, left=1in, right=1in]{geometry}
\usepackage{amssymb}
\usepackage{amsmath}
\usepackage{amssymb}
\usepackage{amsthm}
\usepackage{xfrac}
\usepackage{bm}
\usepackage{graphicx}
\begin{document}
	\begin{center}
		\begin{Huge}
			SPRING 2025 DRP Measures, Curves \& Fractals
		\end{Huge}
		\\[0.1in]
		Aaron Huntley \& Maxwell Seay
		\\[0.1in]
		February 19, 2025
		\\[0.4in]
	\end{center}
    \section{Preface}
    \section{Sketch}
	\subsection{Measure}
	\begin{itemize}
		\item Hausdorff measure
		\item Lebesgue measure
	\end{itemize}
	\subsection{Dimension}
	\begin{itemize}
		\item Hausdorff dimension
		\item Topological dimension
	\end{itemize}
	\subsection{Examples of fractal/fractal-like sets}
	\begin{itemize}
		\item Dragon curves
		\item Osgood curves
		\item Cantor set
		\item DeRham curve
		\item Space-filling curves
	\end{itemize}
	\subsection{Claims/proofs}
	\begin{itemize}
		\item Lebesgue measure of cantor set is 0.
		\item Hausdorff dimension of the cantor set is $\frac{\ln (2)}{\ln (3)}$.
		\item All space filling curves have positive lebesgue measure.
		\item Netto's theorem.
		\item The Hausdorff dimension of a space filling curve is equal to the dimension of its codomain.
		\item Osgood curves are not space filling and don't have Hausdorff dimension 2.
		\item All Osgood curves have positive Lebesgue measure.
	\end{itemize}
	\section{Fractal Construction}
	\section{Notions of Measure}
    \section{Notions of Dimension}
	\subsection{Netto's Theorem}
	If you have a continuous bijection it \textit{can't} hop dimensions.
	\\[0.2in]
	"If one relaxes the requirement of continuity, then all smooth manifolds of bounded dimension have equal cardinality, the cardinality of the continuum. Therefore, there exist discontinuous bijections between any two of them, as Georg Cantor showed in 1878."
\end{document}
