\documentclass[11pt]{ekblite}
\newcommand\mtop{1in}
\newcommand\mbottom{1in}
\newcommand\mleft{1.2in}
\newcommand\mright{1.2in}
\usepackage[top=\mtop, bottom=\mbottom, left=\mleft, right=\mright]{geometry}

\newcommand{\comment}[1]{}
\newcommand\aaron[1]{\textcolor{red}{#1}}
\newcommand{\seay}[1]{\textcolor{blue}{#1}}
\hypersetup{ 	
pdfsubject = {Mathematics},
pdftitle = {fractal dimension},
pdfauthor = {Max Seay and Aaron Huntley}
}

\begin{document}

\title{Exploring fractal dimension\\
DRP spring 2025}
\maketitle
\begin{center}
Maxwell Seay and Aaron Huntley\\
\today
\end{center}

\tableofcontents

\newpage

\section{Preface}
\cite{mandelbrot1}
\cite{falconer1}
\newpage
\section{Notion of Length, Area, Volume $\dots$}
\newpage
\section{Notion of Dimension}
I would like to know how to determine the dimension of arbitrary objects. First I will look at simple examples to find out more about a notion of dimension.
\begin{example}[Line segment]
	I know what a line segment is. I also know every line segment is 1 dimensional.
	\\[0.2in]I take an approximation. I will cover a line segment with boxes. Each box has diameter (or radius) $r$. (To me it doesn't matter much since the difference between radius and diameter is always just a factor of 2.)
	\\[0.2in]Take the line segment to be the interval $[0,1]$. If $r = 1$, then I need only 1 box to cover the line segment.
	\\[0.2in]If $r = \sfrac{1}{2}$, I will need at least 2 boxes.
	\\[0.2in]If $r = \sfrac{1}{3}$, I will need at least 3 boxes.
	\\[0.2in]If $r = \sfrac{1}{n}$, I will need at least $n$ boxes to cover the line segment.
	\\[0.2in]I will define a function that takes in a radius, $r$, and outputs the least number of boxes of radius $r$ needed to cover the line segment. From the above examples it can be seen that this function is:
	\[\mathcal{N}(r) = \frac{1}{r}\]
	Another idea to think on. How to double the \textit{length} of the line segment? To double the length of the line segment, copy and paste it once, and append the copy onto the end of the original. 
	\\[0.2in]So in other words, to double the length of the line segment, add 1, (or $2^1 - 1$ you could say,) extra copy.
\end{example}
\begin{example}[Square]
	I know what a square is. I also know every square is 2 dimensional.
	\\[0.2in]I take an approximation. I will cover a square with boxes. Each box has diameter (or radius) $r$.
	\\[0.2in]Take the square to be the unit square, $[0,1] \times [0,1]$. If $r = 1$, then I need only 1 box to cover the square. (The box and square are the same.)
	\\[0.2in]If $r = \sfrac{1}{2}$, I will need at least 4 boxes.
	\\[0.2in]If $r = \sfrac{1}{3}$, I will need at least 9 boxes.
	\\[0.2in]If $r = \sfrac{1}{n}$, I will need at least $n^2$ boxes to cover the unit square.
	\\[0.2in]I will define a function that takes in a radius, $r$, and outputs the least number of boxes of radius $r$ needed to cover the square. From the above examples it can be seen that this function is:
	\[\mathcal{N}(r) = \frac{1}{r^2}\]
	Now how to double the \textit{area} of the square? To double the area of the square, copy and paste it three times, and append the copies to three sides of the original. 
	\\[0.2in]So in other words, to double the area of the square, add 3, (or $2^2 - 1$ you could say,) extra copies.
\end{example}
So for the line segment we have that the number of squares of radius $r$ needed to cover it as
\[\mathcal{N}(r) = \frac{1}{r}\]
And for the square we have that the number of squares of radius $r$ needed to cover it as
\[\mathcal{N}(r) = \frac{1}{r^2}\]
From these couple examples it seems that, in general
\[\mathcal{N}(r) = c \left(\frac{1}{r}\right)^d\]
It seems that it would be appropriate to call the $d$ in this expression the dimension. This function $\mathcal{N}(r)$ is called a power law since $\sfrac{1}{r}$ is always raised to some power. \cite{falconer2}
\newpage
\section{Cantor Set}
	Let $\mathcal{C}$ be the Cantor Set.
	\begin{definition}[Cantor Set]
        $\mathcal{C}$ can be described as an intersection of an infinite sequence of sets. The sequence of sets, $\{\mathcal{C}_i\}_{i=0}^{\infty}$, is defined recursively. For example, the first three are:
		\[C_0 = [0,1]\]
		\[C_1 = \left[0,\sfrac{1}{3}  \right] \cup \left[ \sfrac{2}{3},1\right]\]
		\[C_2 = \left[0,\sfrac{1}{9}  \right] \cup \left[ \sfrac{2}{9},\sfrac{3}{9} \right] \cup \left[\sfrac{6}{9},\sfrac{7}{9}\right] \cup \left[ \sfrac{8}{9},1\right]\]
		\[C_3 = \left[0,\sfrac{1}{27}  \right] \cup \left[\sfrac{2}{27},\sfrac{3}{27}  \right] \cup \left[\sfrac{6}{27},\sfrac{7}{27}  \right] \cup \left[\sfrac{8}{27},\sfrac{9}{27}  \right] \cup \left[\sfrac{18}{27},\sfrac{19}{27}  \right] \cup \left[\sfrac{20}{27},\sfrac{21}{27}  \right] \cup \left[\sfrac{24}{27},\sfrac{25}{27}  \right] \cup \left[\sfrac{26}{27},1  \right]\]
		\\
		And the recursive definition for a set $\mathcal{C}_i$ is:
		\[\mathcal{C}_i = \frac{1}{3} \mathcal{C}_{i-1} \cup \frac{2}{3} + \frac{1}{3} \mathcal{C}_{i-1}\]
		Or an alternate (non-recursive) definition for a set $\mathcal{C}_i$ is:
		\[\mathcal{C}_i = \bigcup_{j=0}^{i-1} \frac{2j}{3^{i-1}} + \left[j\frac{1}{3^i}, (j+1)\frac{1}{3^i}\right] \cup \left[(j+2)\frac{1}{3^i}, (j+3)\frac{1}{3^i}\right]\]
		(idk I just came up with this. I think it works.)
	\end{definition}
	\begin{corollary}
		$\mathcal{C}_i \ne \mathcal{C}$ for all $i \in \mathbb{N}$. No set in the sequence $\{\mathcal{C}_i\}$ is the Cantor Set, however the further you go (the larger the $i$), the ``closer'' $\mathcal{C}_i$ becomes $\mathcal{C}$. 
	\end{corollary}
	\begin{corollary}
		If $x \notin [0,1]$, then $x \notin \mathcal{C}$.
	\end{corollary}
	\begin{corollary}
		The Cantor Set is non-empty.
	\end{corollary}
	\begin{proof}Consider the point 0.
	\\[0.2in]$0 \in [0,1]$ so $0 \in \mathcal{C}_0$.
	\\[0.2in]Now make the assumption that $0 \in \mathcal{C}_k$.
	\\[0.2in]And since $\frac{1}{3} \cdot 0 = 0$ it must be true that
	\[0 \in \frac{1}{3} \cdot \mathcal{C}_k\] 
	\\[0.2in]Note that by definition we have that
	\[\mathcal{C}_{k+1} = \frac{1}{3} \mathcal{C}_{k} \cup \frac{2}{3} + \frac{1}{3} \mathcal{C}_{k}\]
	Well since $0 \in \frac{1}{3} \cdot \mathcal{C}_k$ it must be true that
	\[0 \in \frac{1}{3} \mathcal{C}_{k} \cup \frac{2}{3} + \frac{1}{3} \mathcal{C}_{k}\]
	or
	\[0 \in \mathcal{C}_{k+1}\]
	So by induction we have that
	\[0 \in \mathcal{C}\]
	Thus $\mathcal{C}$ is non-empty.
	\end{proof}
	\begin{corollary}
		The Cantor Set is a proper subset of $[0,1]$.
	\end{corollary}
	\begin{corollary}
		$\mathcal{C}_{i}$ is always a covering of $\mathcal{C}_{i+1}$ and also always a covering of $\mathcal{C}$.
	\end{corollary}

\newpage
\begin{theorem}[Theorem Example]
        
\end{theorem}

\newpage
\bibliography{citations}

\end{document}
