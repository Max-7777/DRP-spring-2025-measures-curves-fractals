\documentclass[11pt]{ekblite}
\newcommand\mtop{1in}
\newcommand\mbottom{1in}
\newcommand\mleft{1.2in}
\newcommand\mright{1.2in}
\usepackage[top=\mtop, bottom=\mbottom, left=\mleft, right=\mright]{geometry}

\newcommand{\comment}[1]{}
\newcommand\aaron[1]{\textcolor{red}{#1}}
\newcommand{\seay}[1]{\textcolor{blue}{#1}}
\hypersetup{ 	
pdfsubject = {Mathematics},
pdftitle = {fractal dimension},
pdfauthor = {Max Seay and Aaron Huntley}
}

\begin{document}

\title{Exploring fractal dimension\\
DRP spring 2025}
\maketitle
\begin{center}
	Max Seay and Aaron Huntley\\
\today
\end{center}

\tableofcontents

\newpage

\section{Preface}
This was written by Max Seay of CWRU with direction from PhD student Aaron Huntley, as part of the Directed Reading Program in Spring 2025. As of writing, it has not been proofread. Date of latest update on cover page.

\newpage
\section{Introduction}
The goal is to find how to compute the \textit{dimensionality} of some given object. I already know the \textit{dimensionality} of some certain objects. I know the dimension of a point is 0, the dimension of a line is 1, the dimension of a square is 2, etc. But how is dimensionality calculated for some arbitrary object? In this case, by object, I mean:
\\[0.2in]A finite or infinite \textit{compact} set of points $S$, where $S \subset \R^n$.
\\[0.2in]Below are common examples of objects (sets) and their dimensionality.
\begin{itemize}
	\item $\{0\}$ \\[0.1in] dimensionality: 0
	\item $\{1\}$ \\[0.1in] dimensionality: 0
	\item $\{2,4,6,8\}$ \\[0.1in] dimensionality: 0
	\item $[0,1]$ \\[0.1in] dimensionality: 1
	\item $(0,1)$ \\[0.1in] dimensionality: 1
	\item $(0,1]$ \\[0.1in] dimensionality: 1
	\item  $[0,1] \times [0,1]$ \\[0.1in] dimensionality: 2
\end{itemize}

\newpage
\section{How to measure the real numbers}
\begin{figure}[h]
	\includegraphics[scale=0.3]{img/c10.jpg}
	\caption{A few points and line segments living in the real number line $\R$.}
\end{figure}
On the real number line some common objects can be defined. For example:
\begin{example}[Point]
	A point on the real number line is a single real number. Examples include: $1,2,3,17,\sqrt{2},\pi$.
	\\[0.2in]The dimensionality of a point is commonly known to be 0. A point can't have length.
\end{example}
\begin{example}[Line segment]
	A line segment on the real number line is the defined to be the set of all real numbers that lie between two points. A line segment that starts at 0 and reaches 1 would be defined as the set
	\[\{x : 0 \le x \le 1\}\] 
	equivalently expressed in \textit{interval form}:
	\[[0,1]\]
	In other words, a set containing every real number between and including 0 and 1.
	\\[0.2in]The dimensionality of a line is commonly known to be 1. 
\end{example}

\begin{definition}[Interval length]
	Given an interval $I$ of the form $[a,b]$, $(a,b)$, $(a,b]$, or $[a,b)$ where $a \le b$, the length of $I$, notated as $|I|$, is defined to be
	\[|I| := b - a\]
\end{definition}
This is intuitive and useful. But how can this be justified with some definition of measure? And how can this idea of length be extended to a more general way of measurement? 
\subsection{Coverings}
Most definitions of measure and dimension involve the idea of a $\textit{covering}$.
\\[0.2in]A $\textit{covering}$ is some collection of sets that cover some other set. If the covering is made from open sets, it's called an open covering.
\begin{example}[Open covering for an interval]
	Let $I = [1,3]$. An example of an open cover for $I$ is
	\[C = \{(-50,0),(-20,70),(50,100)\}\]
	since 
	\[I \subseteq \bigcup C_i\]
	and every set $C_i$ in $C$ is open.
	\\[0.2in]You could say that $C$ is ``overkill'' as an open cover for $I$ since it spans a much longer length than $I$. In this case, the length of the open cover $\bigcup C_i$ is
	\[100 - (-50) = 150\]
	and the length of $I$ is
	\[3 - 1 = 2\]
\end{example}

\subsection{Infinite sequences of sets}

\begin{example}[Stick cutting]
	Let there be some stick $s$ to measure. Say the stick begins at a length of 1. Now do the following:
	\\[0.2in]Cut the stick to $\frac{1}{2}$ its original length. Then take the stick and cut it to $\frac{1}{3}$ its original length. Then cut the stick to $\frac{1}{4}$ its original length. Continue this forever.
	\\[0.2in]At the end of the cutting process, how long is the stick?
	\\[0.2in]Unfortunately this question doesn't make sense. There is no ``end'' to the cutting process since it goes on forever. First the object should be defined in a valid way. Say $s_0$ is the stick before cutting, $s_1$ is the stick at the first cutting step etc.
	\\[0.2in]A valid definition of the object to be measured is:
	\[\bigcap_{i=0}^{\infty} s_i\]
	In other words, the intersection of every stick. The object to be measured is the set of points that are inside each and every stick $s_i$. Now there is no longer any ambiguity.
	\\[0.2in]Another object which represents this length value is the limit
	\[\lim_{i \rightarrow \infty} |s_i|\]
	\\where $s_i$ = $[0,\frac{1}{i+1}]$ represents the stick after step $i$ of the cutting process.	
	\\[0.2in]In other words, the length of each stick in the sequence as the sequence tends towards infinity.
	\\[0.2in]On one hand, the stick is cut forever, so it must be reduced to nothing by the end. On the other hand, the amount that gets cut off decreases each time and so at some point there may be no more cut off at all.
	\\\includegraphics[scale=0.33]{img/c5.jpg}
	\\\textit{Diagram of the first 9 steps of stick cutting.}
	\begin{corollary}
		For each stick $s_i$ in the sequence of sticks $\{s_i\}_{i=1}^{\infty}$, $s_{i} \subset s_{i-1}$. Each stick $s_i$ is covered by stick $s_{i-1}$. 
		\end{corollary}
	\begin{corollary}
	\[\lim_{i \rightarrow \infty} |s_i| = 0\]
	where $s_i$ = $[0,\frac{1}{i+1}]$ represents the stick after step $i$ of the cutting process.	
	\end{corollary}
	\begin{proof}
		Rewrite the limit with the definition of the length of $s_i$: 
		\[\lim_{i \rightarrow \infty} |s_i| = \lim_{i \rightarrow \infty} \left(\frac{1}{i+1} - 0 \right)\]
		So we have
		\[\lim_{i \rightarrow \infty} |s_i| = \lim_{i \rightarrow \infty} \left(\frac{1}{i+1} - 0 \right)= \lim_{i \rightarrow \infty} \left(\frac{1}{i+1} \right)\]
		This is a common limit known to be
		\[\lim_{i \rightarrow \infty} \left(\frac{1}{i+1} \right) = 0\]
		Thus we have that
		\[\lim_{i \rightarrow \infty} |s_i| = 0\]
		as required.
	\end{proof}
	\begin{corollary}
		The length of $\cap_{i=1}^{\infty} s_i$ is 0. 
	\end{corollary}
	To prove this, I will show that given any real number $\varepsilon > 0$, the length of the stick is less than $\varepsilon$. Thus the length must be 0.
	\begin{proof}
		Let the stick be an interval (line segment). Let $s_0$ be the initial stick; a closed interval $[0,1]$.
		\\[0.2in]Step $1$ is to cut the interval in half. So the interval becomes $\left[0,\frac{1}{2}\right]$.
		\\[0.2in]Step $2$ is to cut the interval into a third. So the interval becomes $\left[0,\frac{1}{3}\right]$.
		\\[0.2in]Step $3$ is to cut the interval into a fourth. So the interval becomes $\left[0,\frac{1}{4}\right]$.
		\\[0.2in]Step $k$ is to cut the interval $\left[0,\frac{1}{k}\right]$ into the interval $\left[0,\frac{1}{k+1}\right]$.
		\\[0.2in]After every step of the cutting process, the stick is cut; the length of the interval is decreased. This can be seen by comparing the lengths of the stick before and after some cutting step $k$:
		\[\left| 0 - \frac{1}{k+1} \right| \le \left| 0 - \frac{1}{k} \right|\] 
		\\[0.2in]Let $\varepsilon \in \R^+$ be some arbitrary positive real number.
		\\[0.2in]I will show that there always exists a cutting step $k'$ such that after this step, the length of the stick is less than the arbitrarily small $\varepsilon$. 
		\\[0.2in]Choose the $k'$ step to be a whole number such that
		\[k' > \frac{1}{\varepsilon}\]
		(By the archimedian principle this is always possible.)
		\\[0.2in]I know that by definition of the cutting process, for this step $k'$, the stick is cut to a length $l$ that is
		\[l \le \frac{1}{k'}\] 
		And note that by the definition of $k'$ we have that
		\[l \le \frac{1}{k'} < \frac{1}{\sfrac{1}{\varepsilon}} = \varepsilon\]
		Or simply
		\[l < \epsilon\]
		Since the length of the stick $l$ at some step $k'$ is always less than $\varepsilon$, and $\varepsilon$ can be made arbitrarily small, it must be true that 
		\[l = 0\] 
		As required.
	\end{proof}
\end{example}
\newpage
\begin{figure}[h]
	\includegraphics[scale=0.25]{img/c6.jpg}
	\caption{The infinite construction of an object living in the real number line $\R$. On each step, each interval is cut and half and the right half is moved towards the right.}
\end{figure}
A common way of measuring an object is to tighly cover it with something. The object's measure should be similar to that of the covering.
\\[0.2in]Like to measure the surface area of a sphere, tighly wrap a blanket around it so that it covers the sphere, then measure the blanket. This covering idea is often a part of definitions of measure and dimension.
\subsection{Lebesgue measure}
Lebesgue measure on the real number line is a method of measuring an object in $\R$ by covering with intervals. The covering of intervals tightly cover the object. The sum of the lengths of intervals is approximately the measure of the object itself.
\begin{definition}[Lebesgue measure]
	
\end{definition}

\newpage
\section{Notion of dimension: Minkowski–Bouligand or box-counting dimension}
I would like to know how to estimate the dimension of arbitrary objects on the plane $\R^2$. First I will look at simple examples to find out more about the notion of dimension in 2D space.
\begin{example}[Line segment]
	I know what a line segment is. I also know every line segment is 1 dimensional.
	\\[0.2in]I take an approximation. I will cover a line segment with boxes. Each box has diameter (or radius) $r$. (To me it doesn't matter much since the difference between radius and diameter is always just a factor of 2.)
	\\[0.2in]Take the line segment to be the interval $[0,1]$. If $r = 1$, then I need only 1 box to cover the line segment.
	\\[0.2in]If $r = \sfrac{1}{2}$, I will need at least 2 boxes.
	\\[0.2in]If $r = \sfrac{1}{3}$, I will need at least 3 boxes.
	\\[0.2in]If $r = \sfrac{1}{n}$, I will need at least $n$ boxes to cover the line segment.
	\\[0.2in]I will define a function that takes in a radius, $r$, and outputs the least number of boxes of radius $r$ needed to cover the line segment. From the above examples it can be seen that this function is:
	\[\mathcal{N}(r) = \frac{1}{r}\]
	Another idea to think on. How to double the \textit{length} of the line segment? To double the length of the line segment, copy and paste it once, and append the copy onto the end of the original. 
	\\[0.2in]So in other words, to double the length of the line segment, add 1, (or $2^1 - 1$ you could say,) extra copy.
\end{example}
\begin{example}[Square]
	I know what a square is. I also know every square is 2 dimensional.
	\\[0.2in]I take an approximation. I will cover a square with boxes. Each box has diameter (or radius) $r$.
	\\[0.2in]Take the square to be the unit square, $[0,1] \times [0,1]$. If $r = 1$, then I need only 1 box to cover the square. (The box and square are the same.)
	\\[0.2in]If $r = \sfrac{1}{2}$, I will need at least 4 boxes.
	\\[0.2in]If $r = \sfrac{1}{3}$, I will need at least 9 boxes.
	\\[0.2in]If $r = \sfrac{1}{n}$, I will need at least $n^2$ boxes to cover the unit square.
	\\[0.2in]I will define a function that takes in a radius, $r$, and outputs the least number of boxes of radius $r$ needed to cover the square. From the above examples it can be seen that this function is:
	\[\mathcal{N}(r) = \frac{1}{r^2}\]
	Now how to double the \textit{area} of the square? To double the area of the square, copy and paste it three times, and append the copies to three sides of the original. 
	\\[0.2in]So in other words, to double the area of the square, add 3, (or $2^2 - 1$ you could say,) extra copies.
\end{example}
So for the line segment we have that the number of squares of radius $r$ needed to cover it as
\[\mathcal{N}(r) = \frac{1}{r}\]
And for the square we have that the number of squares of radius $r$ needed to cover it as
\[\mathcal{N}(r) = \frac{1}{r^2}\]
From these couple examples it seems that, in general
\[\mathcal{N}(r) = c \left(\frac{1}{r}\right)^d\]
Where $c$ is a constant. It seems that it would be appropriate to call the $d$ in this expression the dimension. This function $\mathcal{N}(r)$ is called a power law since $\sfrac{1}{r}$ is always raised to some power. \cite{falconer2}

\newpage
\section{Notion of dimension: Hausdorff measure and dimension}
\begin{definition}[Diameter of a set]
	Let $S$ be a non-empty set living in $\R^n$. The diameter of $S$ is defined to be 
	\[|S| = \sup \{|x - y| : x,y \in S\}\] 
\end{definition}
\begin{figure}[h]
	\includegraphics[scale=0.25]{img/c14.jpg}
	\caption{Diameter of a set $S$ living in $\R^2$.}
\end{figure}
\begin{definition}[$\delta$-cover]
	Let $S$ be a set living in $\R^n$. A set of sets $\{C_i\}$, (where each $C_i$ is itself a set) is said to be a $\delta$-\textit{cover} of $S$ if 
	\[S \subseteq \bigcup_{i=1}^{\infty} A_i\]
	and $|C_i| \le \delta$ for every $C_i$.\cite{falconer1} 
\end{definition}
\begin{definition}[Hausdorff measure]
	Let $S$ be a set living in $\R^n$. First define the function
	\[\mathcal{H}_{\delta}^{d}(S) = \inf \sum_{i=1}^{\infty} |C_i|^d\]
	where the infimum is taken over each and every possible $\delta$-cover $\{C_i\}$ of the set $S$.\cite{falconer1}
	\\[0.2in]In other words, this function picks the "tightest" $\delta$-cover of $S$ and sums the diameter of everything in the covering. The "tightest" $\delta$-covering is the $\delta$-covering with the smallest sum.
	\\[0.2in]Define the \textit{Hausdorff measure} of the set $S$ to be the limit:
	\[\lim_{\delta \rightarrow 0} \mathcal{H}_{\delta}^d(S) \]
	And denote this as simply
	\[\mathcal{H}^d(S)\]
\end{definition}
\begin{figure}[h]
	\includegraphics[scale=0.2]{img/c12.jpg}
	\includegraphics[scale=0.2]{img/c13.jpg}
	\caption{Two valid $\delta$-covers of an object living in $\R^2$. A relatively tight $\delta$-cover (left) compared to a much less tight $\delta$-cover (right). The sum of the diameters of the tight $\delta$-cover (left) is less than the sum of the diameters of the much less tight $\delta$-cover (right). }
\end{figure}
Let's measure some objects using Hausdorff measure.
\begin{example}[Line segment]
	Let $S$ be a line segment living in $\R^2$. Say that
	\[S = [0,1]\]
	The length of $S$ is 1. Though now I would like to compute this using Hausdorff measure. Recall the function
	\[\mathcal{H}_{\delta}^d (S) = \inf \sum_{i=1}^{\infty} |C_i|^d\]
	where the infimum is taken over every possible $\delta$-covers $\{C_i\}$ of $S$.
	\\[0.2in]Ok for now let
	\[\delta = \sfrac{1}{10}\]
	\[d = 1\]
	With these values we now have the function
	\[\mathcal{H}_{\sfrac{1}{10}}^{1} (S) = \inf \sum_{i=1}^{\infty} |C_i|\]
	where the infimum is taken over every possible $\delta$-cover where $\delta = \sfrac{1}{10}$.
	\\[0.2in]In other words, the infimum is now taken over every possible cover $\{C_i\}$ of $S$ where
	\[|C_i| \le \sfrac{1}{10}\]
	for all $C_i$. 
	\begin{figure}[h]
		\includegraphics[scale=0.25]{img/c19.jpg}
		\includegraphics[scale=0.26]{img/c18.jpg}
		\caption{Two ways to cover the line segment [0,1] with a $\delta$-cover where $\delta = \sfrac{1}{10}$. Each orange circle represents an element $C_i \in \{C_i\}$ of the $\delta$-covering. Notice $[0,1] \subseteq \bigcup \{C_i\}$ and $|C_i| \le \sfrac{1}{10}$ for all $C_i$ (as required of a $\delta$-cover.) Let the top image show the cover $\{C_i\}$ and the bottom image show the cover $\{D_i\}$. It can be seen that}
		\[\sum |C_i| < \sum |D_i|\]
	\end{figure}
	\\[0.2in]For simplicity choose a cover like that of the figure (top image.) That is, arrange the cover such that there are 10 balls each with size $\sfrac{1}{10}$ covering the line segment side-by-side.
	\\[0.2in]For each ball in the cover
	\[|C_i| = \sfrac{1}{10}\]
	So we have
	\[\mathcal{H}_{\sfrac{1}{10}}^{1} (S) = \sum_{i=1}^{10} |C_i| = \sum_{i=1}^{10} \sfrac{1}{10} = \left( \frac{1}{10} + \frac{1}{10} + \dots \right) = 1\]
	Now set
	\[\delta = \sfrac{1}{20}\]
	And consider the same configuration of balls for the $\delta$-cover of $S$. That is, arrange 20 balls side-by-side all of diameter $\sfrac{1}{20}$. Now we have that
	\[\mathcal{H}_{\sfrac{1}{20}}^{1} (S) = \sum_{i=1}^{20} |C_i| = \sum_{i=1}^{20} \sfrac{1}{20} = \left( \frac{1}{20} + \frac{1}{20} + \dots \right) = 1\]
	Now using the same kind of covers as before, send $\delta$ to 0. To approach 0, it suffices to let
	\[\delta = \frac{1}{N}\]
	where $N \in \N$ and $\delta \rightarrow 0$ means $N \rightarrow \infty$.
	\\[0.2in]So now observe the limit
	\[\lim_{\delta \rightarrow 0} \mathcal{H}_{\delta}^{1} (S) = \lim_{\delta \rightarrow 0} \sum_{i=1}^{\sfrac{1}{\delta}} |C_i| = \lim_{\delta \rightarrow 0} \sum_{i=1}^{\sfrac{1}{\delta}} \delta\]
	Now replace $\delta$ with its definition from above 
	\[\lim_{\delta \rightarrow 0} \mathcal{H}_{\delta}^{1} (S) = \lim_{N \rightarrow \infty} \sum_{i=1}^{N} \frac{1}{N}\]
	I know for any and all $N \in \N$ that
	\[\sum_{i=1}^{N} \frac{1}{N} = 1\]
	Thus we have that the Hausdorff measure of the set $S$ where $S = [0,1]$ is
	\[\mathcal{H}^1 (S) = \lim_{\delta \rightarrow 0} \mathcal{H}_{\delta}^{1} (S) = \lim_{N \rightarrow \infty} \sum_{i=1}^{N} \frac{1}{N} = 1\]
\end{example}
\newpage
\section{Cantor Set}
	\begin{figure}[h]
		\includegraphics[scale=0.3]{img/c11.jpg}
		\caption{The infinite construction of the Cantor Set in $\R$. On each step, the middle third of each interval is deleted.}
	\end{figure}
	Let $\mathcal{C}$ be the Cantor Set.
	\begin{definition}[Cantor Set]
        $\mathcal{C}$ can be described as an intersection of an infinite sequence of sets. The sequence of sets, $\{\mathcal{C}_i\}_{i=0}^{\infty}$, is defined recursively. For example, the first three are:
		\[C_0 = [0,1]\]
		\[C_1 = \left[0,\sfrac{1}{3}  \right] \cup \left[ \sfrac{2}{3},1\right]\]
		\[C_2 = \left[0,\sfrac{1}{9}  \right] \cup \left[ \sfrac{2}{9},\sfrac{3}{9} \right] \cup \left[\sfrac{6}{9},\sfrac{7}{9}\right] \cup \left[ \sfrac{8}{9},1\right]\]
		\[C_3 = \left[0,\sfrac{1}{27}  \right] \cup \left[\sfrac{2}{27},\sfrac{3}{27}  \right] \cup \left[\sfrac{6}{27},\sfrac{7}{27}  \right] \cup \left[\sfrac{8}{27},\sfrac{9}{27}  \right] \cup \left[\sfrac{18}{27},\sfrac{19}{27}  \right] \cup \left[\sfrac{20}{27},\sfrac{21}{27}  \right] \cup \left[\sfrac{24}{27},\sfrac{25}{27}  \right] \cup \left[\sfrac{26}{27},1  \right]\]
		\\
		And the recursive definition for a set $\mathcal{C}_i$ is:
		\[\mathcal{C}_i = \frac{1}{3}  \mathcal{C}_{i-1} \cup \frac{2}{3} + \frac{1}{3} \mathcal{C}_{i-1}\]
	\end{definition}
	\begin{corollary}
		$\mathcal{C}_i \ne \mathcal{C}$ for all $i \in \mathbb{N}$. No set in the sequence $\{\mathcal{C}_i\}$ is the Cantor Set, however the further you go (the larger the $i$), the ``closer'' $\mathcal{C}_i$ becomes $\mathcal{C}$. 
	\end{corollary}
	\begin{corollary}
		If $x \notin [0,1]$, then $x \notin \mathcal{C}$.
	\end{corollary}
	\begin{corollary}
		The Cantor Set is non-empty.
	\end{corollary}
	\begin{proof}Consider the point 0.
	\\[0.2in]$0 \in [0,1]$ so $0 \in \mathcal{C}_0$.
	\\[0.2in]Now make the assumption that $0 \in \mathcal{C}_k$.
	\\[0.2in]And since $\frac{1}{3} \cdot 0 = 0$ it must be true that
	\[0 \in \frac{1}{3} \cdot \mathcal{C}_k\] 
	\\[0.2in]Note that by definition we have that
	\[\mathcal{C}_{k+1} = \frac{1}{3} \mathcal{C}_{k} \cup \frac{2}{3} + \frac{1}{3} \mathcal{C}_{k}\]
	Well since $0 \in \frac{1}{3} \cdot \mathcal{C}_k$ it must be true that
	\[0 \in \frac{1}{3} \mathcal{C}_{k} \cup \frac{2}{3} + \frac{1}{3} \mathcal{C}_{k}\]
	or
	\[0 \in \mathcal{C}_{k+1}\]
	So by induction we have that
	\[0 \in \mathcal{C}\]
	Thus $\mathcal{C}$ is non-empty.
	\end{proof}
	\begin{corollary}
		$\mathcal{C}_{i}$ is always a covering of $\mathcal{C}_{i+1}$ and also always a covering of $\mathcal{C}$.
	\end{corollary}
	\newpage
	\begin{figure}[h]
		\includegraphics[scale=0.2]{img/c8.jpg}
		\caption{A set visualized in 2D with 10 iterations constructed the same way as the Cantor Set. However the $\frac{1}{3}$ in the Cantor Set definition is replaced with $\frac{1}{2}$ instead.}
	\end{figure}
	\begin{figure}[h]
		\includegraphics[scale=0.2]{img/c7.jpg}
		\caption{A set visualized in 2D with 10 iterations constructed the same way as the Cantor Set. However the $\frac{1}{3}$ in the Cantor Set definition is replaced with $\frac{3}{4}$ instead.}
	\end{figure}
	\begin{figure}[h]
		\includegraphics[scale=0.2]{img/c9.jpg}
		\caption{A set visualized in 2D with 10 iterations constructed the same way as the Cantor Set. However the $\frac{1}{3}$ in the Cantor Set definition is replaced with $\frac{4}{5}$ instead.}
	\end{figure}
\newpage
\section{Dragon}

\newpage
\bibliography{citations}

\end{document}
